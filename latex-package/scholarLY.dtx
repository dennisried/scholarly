\ProvidesPackage{scholarLY}

\RequirePackage{ifthen}
\RequirePackage{enumitem}
\RequirePackage{xstring}
\RequirePackage{keyval}
\RequirePackage{stringstrings}
% for comments
\RequirePackage{verbatim}
% for custom names
\RequirePackage{titlecaps}

% styles:
\def\annType{}
\def\annMeasure{}
\def\annBeat{}
\def\annVoice{}
\def\annAffected{}
\def\annMessage{}
\def\annMessageO{}
\def\annMessageOO{}
% `annStyles` keys to update styles:
\define@key{annStyles}{type}{\def\annType{#1}}
\define@key{annStyles}{measure}{\def\annMeasure{#1}}
\define@key{annStyles}{beat}{\def\annBeat{#1}}
\define@key{annStyles}{voice}{\def\annVoice{#1}}
\define@key{annStyles}{affected}{\def\annAffected{#1}}
\define@key{annStyles}{message}{\def\annMessage{#1}}
\define@key{annStyles}{message-2nd}{\def\annMessageO{#1}}
\define@key{annStyles}{message-3rd}{\def\annMessageOO{#1}}

% DRAFT and FINAL MODE

\def\customdraft{}
% create custom settings for `draft` mode:
\newcommand\annSetDraft[1]{
  \def\customdraft{#1}
}
\def\customfinal{}
% create custom settings for `final` mode:
\newcommand\annSetFinal[1]{
  \def\customfinal{#1}
}
% let custom final settings carry over to draft mode:
\newcommand{\finalfordraft}{\def\customdraft{\customfinal}}

% break annotations or not:
\newcommand{\letAnnBreakornot}{\par\nobreak}
% let break
\newcommand{\annBreakAllow}{\renewcommand{\letAnnBreakornot}{\par}}
% avoid break
\newcommand{\annBreakAvoid}{\renewcommand{\letAnnBreakornot}{\par\nobreak}}
%
% NOTE If \enumoptions (see below) specifies vertical spacing within items, \letAnnBreak
%      is more likely to result in breaks. Otherwise, LaTeX will try to avoid breaking
%      even with \annBreakAllow employed.
%
% apply any of the usual enumerate arguments to the list:
\newcommand{\enumOptions}{}
\newcommand{\setEnumOptions}[1]{\renewcommand{\enumOptions}{#1}}
%
% inline or stacked annotations:
  \newcommand{\skipornot}{}
  \newcommand{\typeSkipornot}{}
\newcommand{\annInline}{%
  \renewcommand{\skipornot}{, }
  \renewcommand{\typeSkipornot}{}}
\newcommand{\annStacked}{%
  \renewcommand{\skipornot}{\letAnnBreakornot}
  \renewcommand{\typeSkipornot}{\letAnnBreakornot}}

% custom type name
\define@key{scholarLYannotations}{type}{\def\lytype{#1}}

\def\prependType{}
\def\appendType{}
\def\annType{}

% hide or show annotation types:
\newcommand{\annHideType}{%
  \def\annName{}
  \renewcommand{\typeSkipornot}{}
}
\newcommand{\annShowType}{%
  \def\annName{{\annType\annTypeName }}
  \ifthenelse{\equal{\skipornot}{\letAnnBreakornot}}
    {\renewcommand{\typeSkipornot}{\letAnnBreakornot}}
    {\renewcommand{\typeSkipornot}{}}
}

\input{default-stylesheet.inp}
\def\annStyleMode{}
\def\annRevisit{}

% PROCESS OPTIONS:
\newif\ifDraft
  \DeclareOption{draft}{\Drafttrue}
  \DeclareOption{final}{\Draftfalse}
\newif\ifDefault
  \DeclareOption{default}{\Defaulttrue}
  \DeclareOption{custom}{\Defaultfalse}
\ProcessOptions*
% use final (implicit) or draft:
\ifDraft
  \def\annStyleMode{\customdraft}
  \def\annRevisit{\revisitDraftorNot}
\else
  \def\annStyleMode{\customfinal}
  \def\annRevisit{\revisitFinalorNot}
\fi
% use custom (implicit) or default
\ifDefault
  \def\revisitFinalorNot{\defaultFinal}
  \def\revisitDraftorNot{\defaultDraft}
\else
  \def\revisitFinalorNot{}
  \def\revisitDraforNot{}
\fi

% ADDITIONAL CUSTOMIZATIONS:
% do something immediately after the type (only effective if type present)
\def\annPostType{}
\define@key{annPrefixes}{post-type}{\def\annPostType{#1}}
% do something immediately before each type or arg (arg presently affects all):
\def\annTypePrePrefix{}
\define@key{annPrefixes}{pre-type}{\def\annTypePrePrefix{#1}}
\def\annArgsPrePrefix{}
\define@key{annPrefixes}{pre-args}{\def\annArgsPrePrefix{#1}}
% first message wrapper (goes inside ann-footnote)
\def\prependMessage{}
\def\appendMessage{}
% default seperator for message wrapper
\def\aMWSep{/}
% macro to redefine the seperator
\newcommand\annMessageWrapSep[1]{\def\aMWSep{#1}}
% macro to define message wrap
\newcommand\annMessageWrap[1]{
  \expandarg\IfSubStr{#1}{\aMWSep}
    {\StrCut{#1}{\aMWSep}\prependMessage\appendMessage}
    {\def\prependMessage{#1}
    \def\appendMessage{#1}}}
% outer message wrap
\def\prependMessageFirst{}
\def\appendMessageLast{}
% default seperator for message outer wrapper
\def\aMWOSep{/}
% macro to redefine the seperator
\newcommand\annMessageWrapOuterSep[1]{\def\aMWOSep{#1}}
% macro to define outer message wrap
\newcommand\annMessageWrapOuter[1]{
  \expandarg\IfSubStr{#1}{\aMWOSep}
    {\StrCut{#1}{\aMWOSep}\prependMessageFirst\appendMessageLast}
    {\def\prependMessageFirst{#1}
    \def\appendMessageLast{#1}}}
% ann type wrapper
\def\prependType{}
\def\appendType{}
% default seperator for type wrapper
\def\aTWSep{/}
% macro to redefine the seperator
\newcommand\annTypeWrapSep[1]{\def\aTWSep{#1}}
% macro to define type wrap
\newcommand\annTypeWrap[1]{
  \expandarg\IfSubStr{#1}{\aTWSep}
    {\StrCut{#1}{\aTWSep}\prependType\appendType}
    {\def\prependType{#1}
    \def\appendType{#1}}}

% \annotations = the main command
\def\annEnumerateOn{
  \def\itemornot{\item}
  \def\begEnumerate{\begin{enumerate}[mode=boxed]\enumOptions}
  \def\endEnumerate{\end{enumerate}}}
\def\annEnumerateOff{
  \def\itemornot{}
  \def\begEnumerate{}
  \def\endEnumerate{}}
% default = enumerate ON
\annEnumerateOn

\newcommand{\annotations}[1]
{\begEnumerate\annStyleMode\annRevisit\input{#1}\endEnumerate}

% keys originally set in arg 1, exported from lilypond
\define@key{scholarLYannotations}{grob}{\def\lygrob{#1}}
\define@key{scholarLYannotations}{grob-location}{\def\lygroblocation{#1}}
\define@key{scholarLYannotations}{grob-type}{\def\lygrobtype{#1}}
\define@key{scholarLYannotations}{input-file-name}{\def\lyinputfilename{#1}}
\define@key{scholarLYannotations}{context-id}{\def\lycontextid{#1}}
\define@key{scholarLYannotations}{location}{\def\lylocation{#1}}
\define@key{scholarLYannotations}{message}{\def\lymessage{#1}}
\define@key{scholarLYannotations}{type}{\def\lyanntype{#1}}
\define@key{scholarLYannotations}{context}{\def\lycontext{#1}}
\define@key{scholarLYannotations}{ann-footnote}{\def\lyannfootnote{#1}}
\def\resetkeys{
  \setkeys{scholarLYannotations}{%
    grob = no value given,
    grob-location = no value given,
    grob-type = no value given,
    input-file-name = no value given,
    context-id = no value given,
    location = no value given,
    type = no value given,
    message = no value given,
    context = no value given,
    ann-footnote = {no value given}
  }
}
\def\noMessError{\color{red} Oops! No message has been entered for this annotation.}

% default Prefixes
\def\annGrobPrx{}
\def\annGrobLocationPfx{}
\def\annGrobTypePfx{}
\def\lyinputfilenamePfx{}
\def\lycontextidPfx{}
\def\annLocationPfx{}
\def\annTypePfx{}
\def\annMessagePfx{}
\def\annContextPfx{}
\def\annMeasurePfx{M.}
\def\annBeatPfx{beat }
\define@key{annPrefixes}{grob}{\def\annGrobPfx{#1}}
\define@key{annPrefixes}{grob-location}{\def\annGrobLocationPfx{#1}}
\define@key{annPrefixes}{grob-type}{\def\annGrobTypePfx{#1}}
\define@key{annPrefixes}{input-file-name}{\def\annInputFileNamePfx{#1}}
\define@key{annPrefixes}{context-id}{\def\annContextIDPfx{#1}}
\define@key{annPrefixes}{location}{\def\annLocationPfx{#1}}
\define@key{annPrefixes}{type}{\def\annTypePfx{#1}}
\define@key{annPrefixes}{message}{\def\annMessagePfx{#1}}
\define@key{annPrefixes}{context}{\def\annContextPfx{#1}}
\define@key{annPrefixes}{measure}{\def\annMeasurePfx{#1}}
\define@key{annPrefixes}{beat}{\def\annBeatPfx{#1}}

% retrieve measure and beat from grob-location
% (TODO: figure out how to address beat fractions)
\def\processGrobLocation{
  \StrBetween[1,1]{\lygroblocation}{(beat-string .}{)}[\lybeatstring]
    \StrDel{\lybeatstring}{ }[\lybeatstring]
  \StrBetween[1,2]{\lygroblocation}{(beat-fraction .}{)}[\lybeatfraction]
    \StrDel{\lybeatfraction}{ }[\lybeatfraction]
  \StrBetween[1,3]{\lygroblocation}{(beat-part .}{)}[\lybeatpart]
    \StrDel{\lybeatpart}{ }[\lybeatpart]
  \StrBetween[1,4]{\lygroblocation}{(our-beat .}{)}[\lyourbeat]
    \StrDel{\lyourbeat}{ }[\lyourbeat]
    \StrDel{\lyourbeat}{ }[\lyourbeatPrev]
  \StrBetween[1,5]{\lygroblocation}{(measure-pos .}{)}[\lymeasurepos]
    \StrDel{\lymeasurepos}{ }[\lymeasurepos]
  \StrBetween[1,6]{\lygroblocation}{(measure-no .}{)}[\lymeasureno]
    \StrDel{\lymeasureno}{ }[\lymeasureno]
  \StrBetween[1,7]{\lygroblocation}{(rhythmic-location }{)}[\lyrhythmiclocation]
}

% if same measure, optionally substitute:
\def\previousMeasure{0}
\def\currentMeasure{0}
% default.. is redefined by key:
\def\sameMeasureStub{---}
\define@key{annExtras}{same-measure}{\def\sameMeasureStub{#1}}
% optionally substitute same beat (if *also* same measure)
\def\previousBeat{0}
\def\currentBeat{0}
% default.. is redefined by key:
\def\sameBeatStub{{---}\unskip}
\define@key{annExtras}{same-beat}{\def\sameBeatStub{#1}}
% to sub same (measure, or measure+beat (never just beat)) when same:
\def\sameLocationLoose{
  \ifthenelse{\equal{\currentBeat}{\previousBeat}}
    {\ifthenelse{\equal{\currentMeasure}{\previousMeasure}}
      {\def\thisMeasure{\sameMeasureStub}
      \def\thisBeat{}}
      {\def\thisMeasure{\annMeasurePfx\currentMeasure,}
      \def\thisBeat{\annBeatPfx\currentBeat}}}
    {\ifthenelse{\equal{\currentMeasure}{\previousMeasure}}
        {\def\thisMeasure{\sameMeasureStub}
        \def\thisBeat{\annBeatPfx\currentBeat}}
        {\def\thisMeasure{\annMeasurePfx\currentMeasure,}
        \def\thisBeat{\annBeatPfx\currentBeat}}}}
% to sub same ONLY when both same:
\def\sameLocationStrict{
  \ifthenelse{\equal{\currentMeasure}{\previousMeasure}}
    {\ifthenelse{\equal{\currentBeat}{\previousBeat}}
      {\def\thisMeasure{\sameMeasureStub}
      \def\thisBeat{\unskip}}
      {\def\thisMeasure{\annMeasurePfx\currentMeasure,}
      \def\thisBeat{\annBeatPfx\currentBeat}}}
    {\def\thisMeasure{\annMeasurePfx\currentMeasure,}
    \def\thisBeat{\annBeatPfx\currentBeat}}}
% to always show both
\def\sameLocationShow{
  \def\thisMeasure{\annMeasurePfx\currentMeasure,}
  \def\thisBeat{\annBeatPfx\currentBeat}}
% macros for show / strict-hide / loose-hide
\def\annSameLocationShow{\def\testLocation{\sameLocationShow}}
\def\annSameLocationStrict{\def\testLocation{\sameLocationStrict}}
\def\annSameLocationLoose{\def\testLocation{\sameLocationLoose}}
% default:
\annSameLocationShow

% The PARSERS:

% determine if there are custom footnotes, by searching arg #1
% (\bigmessage is defined by arg #1 for each annotation)
\def\customFootsParse{
  \StrCount{\bigmessage}{-text=}[\howmanytext]
  \StrCount{\bigmessage}{ann-footnote=}[\howmanyannfn]
  \ifthenelse{\equal{\howmanyannfn}{0}}{
    \ifthenelse{\equal{\howmanytext}{0}}
      {\def\resultOne{}\def\resultTwo{}\def\resultThree{}\def\resultFour{}\def\resultFive{}}
      {\ifthenelse{\equal{\howmanytext}{1}}
        {\StrBehind[1]{\bigmessage}{-text=}[\resultOne]
        \def\resultTwo{}\def\resultThree{}\def\resultFour{}\def\resultFive{}}
        {\ifthenelse{\equal{\howmanytext}{2}}
          {\StrBetween[1,2]{\bigmessage}{-text=}{fn-}[\resultOne]
            \StrGobbleRight{\resultOne}{2}[\resultOne]
          \StrBehind[2]{\bigmessage}{-text=}[\resultTwo]
          \def\resultThree{}\def\resultFour{}\def\resultFive{}}
          {\ifthenelse{\equal{\howmanytext}{3}}
            {\StrBetween[1,2]{\bigmessage}{-text=}{fn-}[\resultOne]
              \StrGobbleRight{\resultOne}{2}[\resultOne]
            \StrBetween[2,3]{\bigmessage}{-text=}{fn-}[\resultTwo]
              \StrGobbleRight{\resultTwo}{2}[\resultTwo]
            \StrBehind[3]{\bigmessage}{-text=}[\resultThree]
            \def\resultFour{}\def\resultFive{}}
            {\ifthenelse{\equal{\howmanytext}{4}}
              {\StrBetween[1,2]{\bigmessage}{-text=}{fn-}[\resultOne]
                \StrGobbleRight{\resultOne}{2}[\resultOne]
              \StrBetween[2,3]{\bigmessage}{-text=}{fn-}[\resultTwo]
                \StrGobbleRight{\resultTwo}{2}[\resultTwo]
              \StrBetween[3,4]{\bigmessage}{-text=}{fn-}[\resultThree]
                \StrGobbleRight{\resultThree}{2}[\resultThree]
              \StrBehind[4]{\bigmessage}{-text=}[\resultFour]
              \def\resultFive{}}
              {\ifthenelse{\equal{\howmanytext}{5}}
                {\StrBetween[1,2]{\bigmessage}{-text=}{fn-}[\resultOne]
                  \StrGobbleRight{\resultOne}{2}[\resultOne]
                \StrBetween[2,3]{\bigmessage}{-text=}{fn-}[\resultTwo]
                  \StrGobbleRight{\resultTwo}{2}[\resultTwo]
                \StrBetween[3,4]{\bigmessage}{-text=}{fn-}[\resultThree]
                  \StrGobbleRight{\resultThree}{2}[\resultThree]
                \StrBetween[4,5]{\bigmessage}{-text=}{fn-}[\resultFour]
                  \StrGobbleRight{\resultFour}{2}[\resultFour]
                \StrBehind[5]{\bigmessage}{-text=}[\resultFive]}
                {}}}}}}}
    {\ifthenelse{\equal{\howmanytext}{0}}
      {\def\resultOne{}\def\resultTwo{}\def\resultThree{}\def\resultFour{}\def\resultFive{}}
      {\ifthenelse{\equal{\howmanytext}{1}}
        {\StrBetween[1,1]{\bigmessage}{-text=}{, ann-footnote}[\resultOne]
        \def\resultTwo{}\def\resultThree{}\def\resultFour{}\def\resultFive{}}
        {\ifthenelse{\equal{\howmanytext}{2}}
          {\StrBetween[1,2]{\bigmessage}{-text=}{fn-}[\resultOne]
          \StrBetween[2,1]{\bigmessage}{-text=}{ann-footnote}[\resultTwo]
          \def\resultThree{}\def\resultFour{}\def\resultFive{}
          {\ifthenelse{\equal{\howmanytext}{3}}
            {\StrBetween[1,2]{\bigmessage}{-text=}{fn-}[\resultOne]
            \StrBetween[2,3]{\bigmessage}{-text=}{fn-}[\resultTwo]
            \StrBetween[3,1]{\bigmessage}{-text=}{ ann-footnote}[\resultThree]
            \def\resultFour{}\def\resultFive{}}
            {\ifthenelse{\equal{\howmanytext}{4}}
              {\StrBetween[1,2]{\bigmessage}{-text=}{, fn-}[\resultOne]
              \StrBetween[2,3]{\bigmessage}{-text=}{, fn-}[\resultTwo]
              \StrBetween[3,4]{\bigmessage}{-text=}{, fn-}[\resultThree]
              \StrBetween[4,1]{\bigmessage}{-text=}{, ann-footnote}[\resultFour]
              \def\resultFive{}}
              {\ifthenelse{\equal{\howmanytext}{5}}
                {\StrBetween[1,2]{\bigmessage}{-text=}{, fn-}[\resultOne]
                \StrBetween[2,3]{\bigmessage}{-text=}{, fn-}[\resultTwo]
                \StrBetween[3,4]{\bigmessage}{-text=}{, fn-}[\resultThree]
                \StrBetween[4,5]{\bigmessage}{-text=}{, fn-}[\resultFour]
                \StrBetween[5,1]{\bigmessage}{-text=}{, ann-footnote}[\resultFive]}
                {}}}}}}}}}
% define the macro names for custom footnotes:
\def\extractFNnames{%
  \expandarg\StrCount{\bigmessage}{fn-}[\theFNnum]
  \ifthenelse{\equal{\theFNnum}{0}}
    {\def\resultOneMacro{}
    \def\resultTwoMacro{}
    \def\resultThreeMacro{}
    \def\resultFourMacro{}
    \def\resultFiveMacro{}}
    {\ifthenelse{\equal{\theFNnum}{1}}
      {\StrBetween{\bigmessage}{fn-}{-text=}[\resultOneMacro]
        \StrDel{\resultOneMacro}{-}[\resultOneMacro]
      \def\resultTwoMacro{}
      \def\resultThreeMacro{}
      \def\resultFourMacro{}
      \def\resultFiveMacro{}}
      {\ifthenelse{\equal{\theFNnum}{2}}
        {\StrBetween{\bigmessage}{fn-}{-text=}[\resultOneMacro]
          \StrDel{\resultOneMacro}{-}[\resultOneMacro]
        \StrBetween[2,2]{\bigmessage}{fn-}{-text=}[\resultTwoMacro]
          \StrDel{\resultTwoMacro}{-}[\resultTwoMacro]
        \def\resultThreeMacro{}
        \def\resultFourMacro{}
        \def\resultFiveMacro{}}
        {\ifthenelse{\equal{\theFNnum}{3}}
          {\StrBetween{\bigmessage}{fn-}{-text=}[\resultOneMacro]
            \StrDel{\resultOneMacro}{-}[\resultOneMacro]
          \StrBetween[2,2]{\bigmessage}{fn-}{-text=}[\resultTwoMacro]
            \StrDel{\resultTwoMacro}{-}[\resultTwoMacro]
          \StrBetween[3,3]{\bigmessage}{fn-}{-text=}[\resultThreeMacro]
            \StrDel{\resultThreeMacro}{-}[\resultThreeMacro]
          \def\resultFourMacro{}
          \def\resultFiveMacro{}}
          {\ifthenelse{\equal{\theFNnum}{4}}
            {\StrBetween{\bigmessage}{fn-}{-text=}[\resultOneMacro]
              \StrDel{\resultOneMacro}{-}[\resultOneMacro]
            \StrBetween[2,2]{\bigmessage}{fn-}{-text=}[\resultTwoMacro]
              \StrDel{\resultTwoMacro}{-}[\resultTwoMacro]
            \StrBetween[3,3]{\bigmessage}{fn-}{-text=}[\resultThreeMacro]
              \StrDel{\resultThreeMacro}{-}[\resultThreeMacro]
            \StrBetween[4,4]{\bigmessage}{fn-}{-text=}[\resultFourMacro]
              \StrDel{\resultFourMacro}{-}[\resultFourMacro]
            \def\resultFiveMacro{}}
            {\ifthenelse{\equal{\theFNnum}{5}}
              {\StrBetween{\bigmessage}{fn-}{-text=}[\resultOneMacro]
                \StrDel{\resultOneMacro}{-}[\resultOneMacro]
              \StrBetween[2,2]{\bigmessage}{fn-}{-text=}[\resultTwoMacro]
                \StrDel{\resultTwoMacro}{-}[\resultTwoMacro]
              \StrBetween[3,3]{\bigmessage}{fn-}{-text=}[\resultThreeMacro]
                \StrDel{\resultThreeMacro}{-}[\resultThreeMacro]
              \StrBetween[4,4]{\bigmessage}{fn-}{-text=}[\resultFourMacro]
                \StrDel{\resultFourMacro}{-}[\resultFourMacro]
              \StrBetween[5,5]{\bigmessage}{fn-}{-text=}[\resultFiveMacro]
                \StrDel{\resultFiveMacro}{-}[\resultFiveMacro]}
              {}}}}}}}
% create the macros that are used in the messages:
\def\makeFNmacros{
  \ifthenelse{\equal{\theFNnum}{0}}
    {}
    {\ifthenelse{\equal{\theFNnum}{1}}
      {\expandafter\def\csname fn\resultOneMacro\endcsname{\footnote{\resultOne} }}
      {\ifthenelse{\equal{\theFNnum}{2}}
        {\expandafter\def\csname fn\resultOneMacro\endcsname{\footnote{\resultOne} }
        \expandafter\def\csname fn\resultTwoMacro\endcsname{\footnote{\resultTwo} }}
        {\ifthenelse{\equal{\theFNnum}{3}}
          {\expandafter\def\csname fn\resultOneMacro\endcsname{\footnote{\resultOne} }
          \expandafter\def\csname fn\resultTwoMacro\endcsname{\footnote{\resultTwo} }
          \expandafter\def\csname fn\resultThreeMacro\endcsname{\footnote{\resultThree} }}
          {\ifthenelse{\equal{\theFNnum}{4}}
            {\expandafter\def\csname fn\resultOneMacro\endcsname{\footnote{\resultOne} }
            \expandafter\def\csname fn\resultTwoMacro\endcsname{\footnote{\resultTwo} }
            \expandafter\def\csname fn\resultThreeMacro\endcsname{\footnote{\resultThree} }
            \expandafter\def\csname fn\resultFourMacro\endcsname{\footnote{\resultFour} }}
            {\ifthenelse{\equal{\theFNnum}{5}}
              {\expandafter\def\csname fn\resultOneMacro\endcsname{\footnote{\resultOne} }
              \expandafter\def\csname fn\resultTwoMacro\endcsname{\footnote{\resultTwo} }
              \expandafter\def\csname fn\resultThreeMacro\endcsname{\footnote{\resultThree} }
              \expandafter\def\csname fn\resultFourMacro\endcsname{\footnote{\resultFour} }
              \expandafter\def\csname fn\resultFiveMacro\endcsname{\footnote{\resultFive} }}
              {}}}}}}}

\def\annMessagePunct{}
% observe the punctuation, if any:
\newcommand{\testMessagePunct}{%
  \IfEndWith{\lymessage}{.}
    {\def\annMessagePunct{.}}
    {\IfEndWith{\lymessage}{?}
      {\def\annMessagePunct{?}}
      {\IfEndWith{\lymessage}{!}
        {\def\annMessagePunct{!}}
        {\def\annMessagePunct{\unskip}}}}}
% parse messages:
\newcommand{\annMessageParse}{%
  \IfEndWith{\lymessage}{.}
    {\StrGobbleRight{\lymessage}{1}[\lymessage]
      \annMessageOO{\annMessageO{\annMessage\prependMessage\lymessage}}\nobreak}
    {\IfEndWith{\lymessage}{?}
      {\StrGobbleRight{\lymessage}{1}[\lymessage]
        \annMessageOO{\annMessageO{\annMessage\prependMessage\lymessage}}\nobreak}
      {\IfEndWith{\lymessage}{!}
        {\StrGobbleRight{\lymessage}{1}[\lymessage]
          \annMessageOO{\annMessageO{\annMessage\prependMessage\lymessage}}\nobreak}
        {\ifthenelse{\equal{\lymessage}{no value given}}{\unskip\noMessError}
          {\ifthenelse{\equal{\lymessage}{}}{\unskip\noMessError}
            {\annMessageOO{\annMessageO{\annMessage\prependMessage\lymessage}}\nobreak}}}}}}
% set the append/punct vs. punct/append orientations:
\newcommand{\annMessTail}{\annMessageOO{\annMessageO{\annMessage\annMessagePunct\appendMessage}}}
% macro to set order: punct, append
\newcommand{\annMessPunctAppend}{%
  \renewcommand{\annMessTail}{
    \annMessageOO{\annMessageO{\annMessage\annMessagePunct\unskip\appendMessage}}}}
% macro to set order: append, punct
\newcommand{\annMessAppendPunct}{%
  \renewcommand{\annMessTail}{%
    \annMessageOO{\annMessageO{\annMessage\appendMessage\unskip\annMessagePunct}}}}

% get type from key
\def\annTypeName{%
  \StrSubstitute{\lyanntype}{-}{ }[\annTypeCleaned]
  \prependType\titlecap{\annTypeCleaned}\appendType}

% store original error handling for unknown keys
\let\KV@errx@ORI\KV@errx

% default annotation parser
\newcommand\annotationParse[1]{
% set KEYS, defaults, then new from #1
  \resetkeys
% ignore unknown keys
    \let\KV@errx\@gobble
    \setkeys{scholarLYannotations}{#1}
% Restore original error handling
    \let\KV@errx\KV@errx@ORI
\saveexpandmode\expandarg
% PARSE FOOTNOTES
  \def\bigmessage{#1}
    \customFootsParse
    \extractFNnames
    \makeFNmacros
% TEST MESSAGE PUNCTUATION, MEASURE and BEAT
  \testMessagePunct
  \processGrobLocation
  \def\currentMeasure{\lymeasureno}
  \def\currentBeat{\lyourbeat}
  \testLocation
% APPLY to ITEM
  \itemornot
    \annTypePrePrefix\annTypePfx\annName\annPostType\typeSkipornot
    {\annArgsPrePrefix\annLocationPfx{\annMeasure{\thisMeasure}}}
    {\annBeat{\thisBeat}\skipornot}
    \ifthenelse{\equal{\lycontextid}{}}
      {\unskip{}}
      {\annArgsPrePrefix\annContextPfx\annVoice{\lycontextid}\skipornot}
    \ifthenelse{\equal{\lygrobtype}{}}
      {\unskip{}}
      {\annArgsPrePrefix\annGrobTypePfx\annAffected{\lygrobtype}\skipornot}
    \annArgsPrePrefix\annMessagePfx{{{%
      \prependMessageFirst\annMessageParse\unskip\annMessTail}\unskip}
      \ifthenelse{\equal{\lyannfootnote}{no value given}}
        {}
        {\unskip{\footnote{\lyannfootnote}}}\appendMessageLast}
% set "previous" mm/beats for use in next annotation;
% arbitrary string parse, to return new value
  \StrDel{\lymeasureno}{ }[\previousMeasure]
  \StrDel{\lyourbeat}{ }[\previousBeat]}

% annotation macros:
\newcommand{\criticalRemark}[1][]{\annotationParse{#1}}
\newcommand{\musicalIssue}[1][]{\annotationParse{#1}}
\newcommand{\lilypondIssue}[1][]{\annotationParse{#1}}
\newcommand{\annotateQuestion}[1][]{\annotationParse{#1}}
\newcommand{\annotateTodo}[1][]{\annotationParse{#1}}
\newcommand{\annotation}[1][]{\annotationParse{#1}}

% custom annotations
% macro to set
\newcommand{\setCustomAnn}[1]{\def\customAnnParse{#1}}
% macro to use
\newcommand\useCustomAnn{\renewcommand\criticalRemark[1][]{\customAnnParse}}

\endinput
